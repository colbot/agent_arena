\documentclass{fdureport}
\pagestyle{plain}

% ref style
\usepackage[nameinlink]{cleveref}
\crefformat{equation}{#2\bfseries{公式#3}~#1#2#3} % 方程式引用格式
\crefformat{figure}{#2\bfseries{图#3}~#1#2#3}     % 图表引用格式
\crefformat{table}{#2\bfseries{表#3}~#1#2#3}      % 表格引用格式

% code style
\usepackage{listings}
\lstset{
    numbers=left,
    numberstyle=\tiny\color{lightgray},
    backgroundcolor=\color{lightgray!20},
    keywordstyle= \color{ blue!70}\bfseries,%设置关键字颜色
    commentstyle=\color{green!40!black}, %设置注释颜色
    stringstyle=\color{red}, % 字符串颜色
    frame=single, % 阴影效果
    rulesepcolor= \color{ red!20!green!20!blue!20} ,
    escapeinside=``, % 英文分号中可写入中文
    escapechar=`,%设置转义字符为反引号
    basicstyle=\linespread{1.0} \ttfamily \small,
    breaklines=true,%在单词边界处换行。
    showstringspaces=false, %去掉空格时产生的下划的空格标志, 设置为true则出现
    columns=fullflexible,%可以自动换行
    % xleftmargin=2em, %距离左边界2em
    % aboveskip=1em,
    % framexleftmargin=2em,
    % linewidth=1\linewidth, %设置代码块与行同宽
    % breakatwhitespace=ture,%可以在空格处换行
}

% list style
\usepackage{enumitem}
\setenumerate[1]{itemsep=0pt, parsep=\parskip, partopsep=0pt, topsep=5pt}

\begin{document}

\title{基于腾讯开悟教学平台采用PPO算法进行的智能体决策1V1实验}
\lesson{神经网络和深度学习}
\teacher{赵卫东}
\expgroup{DayDayUp}
\expmembers{张广东 24262010055,余盛龙 24262010053,许丽瑄 24262010046,甘成武 24262010010,黄正宇 24262010018}

\makecover

{
\linespread{1.2} \selectfont
\setcounter{tocdepth}{2}
\tableofcontents
}
\newpage

\graphicspath{{figure}}
% section 1 
\include*{section/exp_target}

% section 2
\include*{section/exp_frame}

% section 3
\include*{section/exp_design}

\include*{section/exp_process}

\include*{section/exp_conclusion}

\end{document}